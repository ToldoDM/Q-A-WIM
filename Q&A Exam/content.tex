\section{Q\&A WIM}
\subsection{Body Spam: uses, advantages and drawbacks}
The Body Spam is a technique used to increase
the score of a web page calculated by search
engines.\\
It consists in inserting the keywords directly
in the body of the document.\\
Some advantages are that it is simple and effective
but compromises have to be found due to the TF*IDF
tecnique used by search engines to counterbalance the value of each keyword.
\\
Some disadvantages are that the page content is touched and for this reason
keywords has to be put in a proper way in order not to unsatisfy
the user.
So it is also important to provide a sensible
and well-structured page to the user, hiding the keywords inserted
in the body.\\
This can be done by inserting the text in white on a
white background or by redirecting the user to
the page we want him to view. However, in the last case, it is necessary
to do it with a Javascript code, so that the search engines would
skip the code because it requires so much time to be analyzed.\\
Another technique, which is not permitted and it is considered
ethically incorrect, is the cloaking.
It consists in providing different pages under the same address to the
search engines bots.\\
If it is the crawler, it is shown the page with the keywords.\\
If it is the users, it is shown the page they want to see.\\
It is the most effective and the most difficult to detect
(an employee should come and check the page in person)
but it leads to greater penalties in
case of being discovered, such as, for example, long-term
suspension from indexing and search results.

\subsection{Title spam: uses, advantages and drawbacks}
The Title Spam is a technique used to increase
the score of a web page calculated by search
engines.\\
It consists in inserting the keywords directly
in the title of the document.\\
Some advantages are that it is simple and effective
but compromises have to be found due to the TF*IDF
tecnique used by search engines to counterbalance the value of each
keyword.\\ Some disadvantages are that the page content is touched and for this reason
keywords has to be put in a proper way in order not to unsatisfy
the user. However it touches less the page content and it is something
which is less visible by the user because it is outside the content
of the page itself.

\subsection{Meta tag spam: uses, advantages and drawbacks}
The Meta tag spam is a technique used to increase the score of a web page calculated
by search engines.\\
It consists in entering the keywords in the
appropriate meta-data tag. The advantage is that the content of
the page is not affected.\\
The disadvantage is that search engines tend not to give much weight
to the words entered there, especially if they are repeated several times
because it is an simple and highly used technique.

\subsection{Anchor text spam: uses, advantages and drawbacks}
The anchor text spam is a technique used to increase the score of a web page calculated
by search engines.\\
It consists in inserting the keywords in the names of the links to
other pages. In fact the anchors are part of the body of a page but are
treated separately.\\
Here some advantages. Keywords become very visible because links are
with different colours and underlined.\\
Search engines usually tend to give a high
weight to the terms inserted in the anchors.\\ Furthermore, the keywords
inserted in the anchors are usually also added to the target pages
without being subjected to penalty filters. This happens because an
anchor is supposed to give an idea of the page it points to.\\
A disadvantage can be that it can touch the real content of the pages, so
it is important not to put unrelated content in order not to get the user
angry.

\subsection{URL spam: uses, advantages and drawbacks}
The URL spam is a technique used to increase the score of a web page calculated
by search engines.\\
It consists in inserting the keywords directly in the web address
of the page. In fact search engines index and use the URLs itself
to calculate the scores of the pages, giving bonuses similar to those
of anchortext spam.\\
The advantage is that it does not affect the content of the page
and is usually used in combination with other term spam techniques.\\
Even with this technique it is necessary to act rationally because as disadvantage
there is the fact that
if you insert the same word multiple times you will be penalized by TF-IDF.

\subsection{Repetition technique: uses, advantages and drawbacks}
Repetition is a technique used to increase the keyword score on a web page.\\
It consists in repeating one or a few keywords several times.\\
In this way
it generates the advantage of
increasing the relevance of the page with respect to a single one
or to a low number of keywords.\\ There are also some disadvantages.
Because it is a technique that is very easily
detectable by crawlers, it is necessary, if you do not want to be penalised,
to pay particular attention to countermeasures.
Another thing to take into consideration is the TF*IDF in order not to be
penalised because some keywords are used so many times.

\newpage

\subsection{Dumping technique: uses, advantages and drawbacks}
The Dumping is a technique used to increase the score of a
web page calculated by search engines.\\
It consists in inserting a large number of rare terms, even if they
are not related to the content of the page. Here some advantages.
The page will be relevant for many different terms which will have only
a few other relevant pages because they are considered rare.\\
But there are also disadvantages.
Inserting terms not relevant to the content on the page makes more difficult
keeping user, even if it facilitates access to the site.
This because what our page offers will be probably different from what
they were looking for, leading also loss of trust by the user.

\subsection{Weaving technique: uses, advantages and drawbacks}
The Weaving is a technique used to increase the score of a web page
calculated by search engines.\\
It consists in copying text from other web pages and inserting keywords
in random positions.
This technique works best if the topic covered by the copied text is rare
or for which there are only a few relevant pages.\\
There are also some other advantages. It is also used to utilyze better
the keywords entered,
so that they can be repeated several times, reducing the possibility
of being penalised by TF-IDF.\\ It is an automatic way for making
interesting content and attracting more the users.\\
As disadvantage, it affects the content of the page, so
it is important not to put unrelated content in order not to get the user
angry.

\subsection{Stitching technique: uses, advantages and drawbacks}
The Stitching is a technique used to increase the score of a web page
calculated by search engines.\\ It consists in paste\&copying from
different web sources and then assembling everything with the aim
of quickly obtaining relevant content.\\
Here there are some advantages. This technique is useful for quickly
populating a site and for making the page have the possibility of be
shown in the search results for each of the topics covered in the copied
texts. Another benefit of this technique is that many search engines
reward sites with more pages.

\subsection{Broadening technique: uses, advantages and drawbacks}
The Broadening is a technique used to increase the score of a web page
calculated by search engines.\\
It consists in entering synonyms and related phrases in addition to
entering the chosen keywords, also.
Here some advantages. It is seen very
positively by search engines.
In particular this helps to cover more user queries because they are
not always very precise. There can be also some extra bonuses if the
keywords are similars to each others.\\
As disadvantage, it may be a good thing to control the penalisation made
by FT-IDF. In fact, choosing different words can also lead to a decrease of
ranking.

\subsection{Describe the LOD classification: its features, uses and potential advantages and drawbacks}
Linked Data are a generalization of the semantic web and they are used to
create knowledge graphs.\\
LOD, linked open data, are linked data that can be used free.
It is very important to facilitate their creation and management because they can produce some advantage such as
making new knowledge and facilitate innovation, offering the possibility
to produce better services and applications.\\
LOD are classified from 1 to 5 stars:
\begin{enumerate}
    \item The information is available on the web, in any format, under an open license (free to use).
    \\(example: Images)
    \item The same as above with data in a machine-readable structured format.
    \\(example: Excel)
    \item The same as above and using a non-proprietary structured format.
    \\(example: CSV)
    \item The same as above and in semantic web format,
    therefore URI identifiers are used so that it is possible to point to
    a single piece of data.
    \\(example: RDF, OWL)
    \item The same as above and with data are linked to other sources to provide context\\
    (example: Graph)
\end{enumerate}
Another advantage is that, starting from 3-star data, it is possible to reach 4 or 5 stars by
giving it a semantic format.\\
This operation is called a lifting operation and there are also tools
to do this. This is done by joining knowledge graphs and using
similarity measures in order to see which graphs can be connected.\\
The opposite operation is called the lowering operation.\\
Both operations are called mashup, which consist in mixing
RDF with structured data.\\
The main problem of LOD is that it is necessary to use big data
analysis algorithms for processing these data. It is also a
computational expensive process and the user does not always feel
to be able to interface with a database to extract the data he needs.\\
Concrete examples of LOD are DBpedia and schema.org.

\newpage

\subsection{RDF and RDFS: their features, uses, advantages and drawbacks.}
RDF (Resource Description Framework) is a technology that allows
to structure information and remove ambiguities.\\
It provides data semantics and allows machines to understand the
information present on the web.\\
RDF is written with N-triplets instead of XML because of the existing
dialects of XML which make information aggregation impossible.
This technology allows to describe metadata, relationships and concepts
ensuring interoperability between them (RDF advantage).\\
The information description is
based on the basic grammar composed by triple subject, predicate,
direct object and each of these three elements can contain strings or URIs.
This structure can be visualized as a graph called knowledge graph
and the RDF allows the aggregation of multiple knowledge graphs
by linking resources identified by the same URI (RDF advantage).\\
However, the objects must also be classified ensuring a minimum
computational cost and this is allowed by the RDF Schema.\\
The RDFS is a schema with an information structure which is made
of classes, sub-classes and individuals for objects,
while is made of
properties, sub-properties, domains and intervals (range) for verbs.\\
This implies more power for integrity checks and deductions (RDFS advantage).\\
The RDF has established basic level semantic rules and so
the RDF-Schema allows to define ontologies to categorize information
taxonomically (RDFS advantage). This is useful and gives more power
to face the URI variants problem (one concept can be expressed differently).
\\ The main advantages of RDF is that it is well specified
and allows data to be decentralized and distributed, so that anyone
can create a vocabulary or publish data about other
resources. Moreover knowledge graph is conceptually simple to understand
and analyze.
\\Disadvantages of RDF is that it is very abstract and verbose,
so it is difficult to write or read manually. Moreover programming
RDFS requires a knowledge of basic details such as what is a URI
and what is a triple.\\
Two examples of RDF vocabularies
are Dublin Core (DC) and Friend Of A Friend (FOAF).

\subsection{Dublin Core: its uses, advantages and drawbacks}
The Dublin Core is one of the first attempts to structure the web
in a semantic way.\\
It has got 15 basic and optional metadata elements have been defined.
They can be repeatable and independent from the domain in which they
are used (Title, Author, Subject, ID, Source, Date, Language, etc...).\\
An advantage can be that it allows to define the fundamental and
essential basic properties for the entire semantic data structure.\\
A disadvantage can be that it is too generic to adequately describe
specific resources.

\subsection{FOAF: its uses, advantages and drawbacks}
FOAF is a machine-readable ontology describing persons,
their activities and their relations to other people and objects.
It allows groups of people to describe social networks without
the need for a centralised database.\\
Being an RDF application, it can be easily aggregated and combined
with other vocabularies allowing to capture a very rich set of metadata.